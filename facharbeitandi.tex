\documentstyle[12pt,german,refman]{article}

\setleftmarginwidth{2in}

\begin{document}

\pagestyle{myfootings}
\markboth{}{Die Trojaner}

\title{">Tanz der Troianer"<\\
	Facharbeit aus der Physik}
\author{Andreas W"url\\
	Am Galgenfeld 29\\
	8831 H"ottingen\\[0.3cm]
	\today}

\maketitle
\makeauthor

\tableofcontents


\section{Einf"uhrung}
\subsection{Kleine Planeten (Asteroiden, Planetoiden)}

Bereits sehr fr"uh fiel es den Astronomen auf, da"s die Abst"ande der einzelnen
Planeten von der Sonne einer gewissen Ge\-setz\-m"a\-sig\-keit unterliegen. Je weiter
sie vom der Sonne, dem Zentralgestirn unseres Planetensystems, entfernt sind, desto
gr"o"ser werden die L"ucken zwischen den Planetenbahnen. Christian Wolf
\footnote{$\ast$ ??.??.1???, $\dagger$ ??.??.1???} wies im Jahre 1741 als erster auf
diese Merkw"urdigkeit hin, f"ur die sp"ater, im Jahre 1772 von den Astronomen Johann
Daniel Titius und Johan Elert Bode ein mathematisch gefa"stes Gesetz gefunden wurde. Auch der etwas sp"ater, im Jahre 1781 entdeckte Planet
Der 1781 von Wilhelm Herrschel entdeckte Planet Uranus pa"ste au"serordentlich
gut in das Schema. Nur an einer Stelle klaffte eine L"ucke: Nach Titius und Bode
m"u"ste zwischen Mars und Jupiter in einem Sonnenabstand von $\approx 2,8$
AE\footnote{AE = astronomische Einheit; $1 AE = 1,495978\cdot10^{11}m$(L"ange der
gro\3en Halbachse der Erdbahn} noch ein weiterer, bisher nicht entdeckter Planet
seine Runde ziehen. Bisher hatte aber noch nichts auf die Existenz eines solchen
Planeten higewiesen. Sollte er vielleicht doch existieren und wegen seiner
Lichtschw"ache aufgrund der geringen Gr"o\3e noch nicht aufgefallen sein?

Ende des 18.~Jahrhunderts schlossen sich einige Astronomen und Sternwarten zusammen
und begaben sich, im Vertrauen auf die Richtigkeit der Titius-Bodeschen Gesetze, auf
die Suche nach dem bisher nicht entdeckten Planeten zwischen Mars und Jupiter. In
Palermo war um diese Zeit der M"onch Giuseppe Piazzi damit besch"aftigt, einen
Sternkatalog herzustellen. Piazzi wollte sich von der Richtigkeit seiner Angaben
"uberzeugen und so fand er bei seinen Sternbeobachtungen in der Neujahrsnacht
1800/1801, zu seiner gro"sen Verwunderung ein weiteres Sternchen, dessen Helligkeit
nur 8. Gr"o"se war. Bald stellte sich heraus, da"s dies kein Fixstern sein konnte,
denn der Lichtpunkt bewegte sich weiter. Piazzi hatte einen neuen Planet entdeckt,
aber er war sich noch nicht sicher, in welchem Abstand zur Sonne sich dieser
Himmelsk"orper bewegen w"urde. Au"serdem verschwand der Planet am 11. Februar im
Strahlenbereich der Sonne, und es bestand kaum mehr Hoffnung, ihn jemals wieder
aufzufinden. Die wenigen Beobachtungen Piazzis reichten nicht aus, mit den
damaligen Mitteln der Mathematik die Bahn des Wandelsterns zu berechnen.

Piazzi ver"offentlichte seine Beobachtungen und der vierundzwanzigj"ahrige Mathematiker
Carl Friedrich Gau\3 erfuhr so von dessen Problem. Nachdem der franz"osische
Mathematiker Laplace eine Berechnung der Bahn des neuentdeckten Planeten f"ur
unm"oglich erachtet hatte, gelang Gau\3 ei\-ne neue Methode der Bahnbestimmung zu
entwickeln, nach der bis auf den heutigen Tag die Bahnen neuentdeckter Kometen oder
Planeten berechnet werden. Nach seinen Vorausberechnungen konnte der Gothaer Astronom
Zach am 7. Dezember 1801 den nun \glqq Ceres\grqq\ genannten Planeten tats"achlich
wiederaufzufinden. 

Nach und nach wurden immer mehr solcher Kleink"orper, die sogenannten {\it kleinen
Planeten}, {\it Asteroiden} oder {\it Planetoiden} entdeckt. Die Sch"atzunge "uber
die Gesamtmasse der {\it Asteroiden} reichen von ca. \dots bis ca. \dots. Bis zum
Ende des 19. Jahrhunderts wurden weitere hundert solcher {\it Asteroide} gefunden.
Erst als durch Max Wolf in Heidelberg 1890 die Himmelsphotographie zur Suche nach
diesen Himmelsk"orpern eingesetzt wurde, wuchs ihre Anzahl schnell. Zur Jahreswende
1962/63 waren schon 1651 Asteroiden gefunden und katalogisiert.

\subsection{Kleinplaneten und Himmelsmechanik}

Die St"orungstheorie der Himmelsmechanik fordert, da\3 die Umlaufzeiten zweier
Planeten nicht kommensurabel, d.h. nicht im Verh"altnis kleiner ganzer Zahlen
zueinander stehen d"ur\-fen. Wenn anf"anglich die Umlaufzeiten in einem ganzzahligen
Ver"altnis zueinander gestanden haben sollten, dann haben die Gravitationsst"orungen
des gr"o\3eren Planeten auf die Kleinen Planeten diese Kommensurabilit"at beseitigt.
So hat der gr"o\3te aller Planeten des Sonnensystems, der Jupiter, die Schar der
Asteroiden so geordnet, wie wir sie heute vorfinden. 

Seltsamerweise halten sich einige der Planetoiden aber nicht an die St"orungstheorie
der Himmelsmechanik, ihre Umlaufzeiten stehen tats"achlich im Verhaltnis kleiner
ganzer Zahlen zu der Umlaufzeit des Jupiter. Man hat diese Planetoiden nach ihrer
Resonanz\footnote{Die Resonanz entspricht dem Verh"altnis der beiden Umlaufzeiten} in
verschiedene Gruppen eingeteilt:   
\begin{itemize}
	\item{Die Trojaner (21 Stck.,Resonanz $\frac11$), siehe Tab.~\ref{tab:trojaner}}
	\item{Die Hilda--Gruppe (27 Stck., Resonanz $\frac23$)}
	\item{Die \glqq Ungarischen\grqq\ (16 Stck., Resonanz $\frac14$)}
	\item{Die Apollo--Amor Objekte (48 Stck.)}
\end{itemize}
Das Ph"anomen der Trojaner soll im folgenden genauer untersucht werden.

\section{Die Trojaner}
Die Planetoiden der Trojanergruppe (siehe Tabelle~\ref{tab:trojaner}) kreisen zusammen
mit dem Gasriesen Jupiter um die Sonne; ein Teil davon, die {\it Achilles-Gruppe},
eilt um ca. $60^\circ$ voraus, der andere Teil, die {\it Patroclus-Gruppe}, folgt
Jupiter auf seiner Bahn mit einen Abstand von ca. $60^\circ$.    

\begin{table} \centering \begin{tabular}{llllllll} \multicolumn{1}{l}{Nr.} &
\multicolumn{1}{l}{Name} & \multicolumn{1}{l}{mag} & \multicolumn{1}{l}{i} &
\multicolumn{1}{l}{e} & \multicolumn{1}{l}{$ \mu $} & \multicolumn{1}{l}{r} &
\multicolumn{1}{l}{$ \varrho $} \\ \cline{1-8} 
588	& Achilles 	& 16.0	&10.3	& 0.15	& 298	&$+44^o$& 55 km \\
617	& Patroclus	& 15.8	& 22.1	&0.14	& 299	& -63	& 60	\\
624	& Hector 	& 15.2	& 18.3	& 0.02	& 306	& +70	& 78	\\
659	& Nestor	& 16.3	& 4.5 & 0.11	& 296	& +74	& 48	\\
884	& Priamus	& 16.5	& 8.9	& 0.12	& 298 	& -82	& 45	\\
911	& Agamemnon	& 15.4	& 21.9	& 0.07	& 305	& +69	& 70 	\\
1143	& Odysseus	& 16.0	& 3.1	& 0.09	& 300	& +66	& 55	\\
1172 	& Aeneas	& 16.0	&16.7	& 0.10	& 300	& -76	& 55	\\
1173	& Anchises 	& 16.6	& 7.0	& 0.14	& 308	& -45	& 40	\\
1208 	& Troilus	& 16.3	& 33.7	& 0.09	& 303	& -60	& 47	\\
1404	& Ajax	 	& 16.8	& 18.1	& 0.11	& 302	& +85	& 37	\\
1437	& Diomedes 	& 15.8	& 20.6	& 0.04	& 304	& +45	& 61	\\
1583 	& Antilochus	& 16.5	& 28.3	& 0.05	& 293	& +33	& 47	\\
1647 	& Menelaus	& 18.5	& 5.6 & 0.03 	& 299	& +71	& 18	\\
\end{tabular} \caption{Die Trojaner} \label{tab:trojaner} \end{table}

\subsection{Das Dreik"orperproblem von Lagrange}
Schon lange vor der Entdeckung der Kleinplaneten machte sich der franz"osische
Mathematiker Joseph Luis Lagrange Gedanken "uber die Wechselwirkungen dreier K"orper
im Raum; er beschrieb einen Sonderfall des Dreik"orperproblems, bei dem zwei K"orper
durch ihre Massen eine merkliche Schwerkraft aus"uben und ein dritter fast masseloser
K"orper im Schwerefeld der anderen K"orper treibt. Er beweist die Existenz
kr"aftefreier Punkte, sog. Librationspunkte, an denen sich die auftretenden
Gravitations- und Fliehkr"afte aufheben. An solchen Librationspunkten k"onnen also
K"orper kleiner Masse f"ur l"angere Zeit kr"aftefrei verweilen.
Der geometrische Ort dieser Librationspunkte in einem System, das aus zwei um ihren
gemeinsamen Schwerpunkt $S$ rotierenden K"orpern $M_1$ und $M_2$ besteht, ist im
wesentlichen konstant. Drei Librationspunkte liegen auf der die zwei K"orper
verbindenden Geraden ($L_1, L_2, L_3$ im Bild~\ref{pic:Librationspunkte}) zwei
weitere bilden mit den K"orpern gleichschenklige Dreiecke. Wird ein Massepunkt in
einem dieser Punkte mit der richtigen Geschwindigkeit positioniert, so befindet er
sich im Gleichgewichtszustand. Allerdings wird er, wenn er sich nur ein vom
Lagrangepunkt entfernt, von den auftretenden Kr"aften noch weiter weggezogen!
Die restlichen zwei Librationspunkte $L_4$ und $L_5$ bilden mit den beiden Massen
$M_1$ und $M_2$ ein gleichseitiges Dreieck. An diesen Punkten wird der K"orper, wenn
er etwas von seinem Lagrangeschen Punkt abkommt von den entstehenden Kr"aften wieder
zur"uckgezogen.

Das Verhalten der Trojaner la\3t sich mit Hilfe dieser Theorie erkl"aren. Die beiden
Massereichen K"orper, die um ihren gemeinsamen Schwerpunkt kreisen, sind Sonne und
Jupiter. Die Trojaner, im Vergleich dazu fast masselos, schwingen um die
Librationspunkte $L_4$ und $L_5$, die mit Sonne und Jupiter jeweils ein
gleichseitiges Dreieck bilden. Die Schwingung kommt dadurch zustande, da\3 die
K"orper, wenn sie sich vom Lagrangepunkt entfernen, wieder durch die entstehenden
Kr"afte zur"uckgezogen werden.  

\begin{figure}
\centering
\unitlength 1cm
\picture(8,4)
\put(0,2){\line(8,0){8}}
\put(3,2){\line(1,1){2}}
\put(3,2){\line(1,-1){2}}
\put(5,4){\line(1,-1){2}}
\put(5,0){\line(1,1){2}}
\put(3,2){\circle{0.3}}
\put(7,2){\circle{0.2}}
\put(4,2){\circle{0.1}}
\put(0,2){\circle{0.1}}
\put(6,2){\circle{0.1}}
\put(8,2){\circle{0.1}}
\put(5,0){\circle{0.1}}
\put(5,4){\circle{0.1}}
\put(2.5,1.5){$M_1$}
\put(4,1.5){$S$}
\put(7,1.5){$M_2$}
\put(0,1.5){$L_1$}
\put(6,1.5){$L_2$}
\put(8,1.5){$L_3$}
\put(5,3.5){$L_4$}
\put(5,0.5){$L_5$}
\endpicture
\caption{Die Lage der f"unf Librationspunkte}
\label{pic:Librationspunkte}
\end{figure}

\subsection{Die Berechnung der Gleichgewichtsl"osungen}
 Zur L"osung des Problems betrachtet man die Bewegung eines Massepunktes im
Gravitationsfeld zweier K"orper gro"ser Masse ($M_1, M_2$), mit dem Massenverh"altnis
$c=\frac{M_1}{M_2}$, z.B. Sonne und Jupiter im Abstand R, die ihren gemeinsamen
Schwerpunkt $S$ auf elliptischen Bahnen umlaufen. Berechnet man {\it kinetische} und
{\it potetielle Energie}, dieses Massepunktes in korotierenden
Koordinaten\footnote{dieses entspricht einem mitrotierenden Koordinatensystem}, so
kann man nach der Lagrange-Methode drei Differentialgleichungen zweiter Ordnung
aufstellen, als deren Gleichgewichtsl"osungen man die Librationspunkte findet.
Physikalisch gesehen sind diese L"osungen f"ur den Fall von kreis\-f"or\-mi\-gen
Bahnen station"ar, w"ahrend sich f"ur elliptische Bahnen, die Abst"ande der
Librationspunkte von $M_1$ und $M_2$ periodisch "andern.

\section{Berechnung eines anderen Lagrangepunktes}
Zur Veranschaulichung des Effektes von Lagrange werden wir nun ein einfaches Beispiel
verwenden:
Wir betrachten ein System Erde--Mond--Satellit, wobei Erde und Mond die beiden Massen
$M_1$ und $M_2$ darstellen, die Masse des Satelliten ist im Verh"altnis zu Erde und
Mond vernachl"assigbar klein.

\subsection{Dimensionsloses Rechnen}
Zur weiteren Vereinfachung des Programmlaufes setzen wir unsere Gr"o\3en so, da\3
eine Verwendung von nicht ganzzahligen Gr"o\3en, wie z.B. die Naturkonstanten $G$ und
die genauen Massewerte von Erde und Mond, nicht notwendig wird. 

Die Vereinfachungen sehen folgenderma"sen aus:
\begin{enumerate}
\item{\bf Die Entfernungen:} Es wird festgesetzt, da\3 die Entfernung Erde--Mond den Wert
\glq 1\grq\ betr"agt. Die Entfernung Erde - gemeinsamer Schwerpunkt bekommt die
Bezeichung $c$ und die Entfernung Mond - gemeinsamer Schwerpunkt ist somit $1-c$.
\item{\bf Die Massen:} Wenn man die Zentripetalkr"afte f"ur Erde und Mond berechnet und
die beiden entstehenden Gleichungen dann dividiert bekommt man eine Beziehung
zwischen den Massen der Erde und des Mondes.
$$F_{Z,Erde}=M_{Erde}\cdot\omega^2\cdot c;$$
$$F_{Z,Mond}=M_{Mond}\cdot\omega^2\cdot (1-c);$$
$$\frac{F_{Z,Erde}}{F_{Z,Mond}}\longmapsto{M_{Erde}\over M_{Mond}}\cdot\frac{c}{1-c}=1
\Longrightarrow \frac{M_{Erde}}{M_{Mond}}=\frac{1-c}{c};$$
Zur weiteren Vereinfachung kann man nun f"ur die Masse der Erde und des Mondes
folgendes einsetzen: $M_{Erde}=1-c$ und $M_{Mond}=c$.
\item{\bf Die Winkelgeschwindigkeit:}Wenn wir die bisher vereinfachten Werte zur
Hilfe nehmen, k"onnen wir die Zentripetalkraft $F_Z$ und die Gravitationskraft $F_G$
gleichsetzen.
$$F_{Z,Erde}=F_{Z,Mond}=F_G$$
$$(1-c)\cdot\omega^2\cdot c=(1-c)\cdot c\Longrightarrow \omega^2=1;$$
Somit hat die Winkelgeschwindigkeit $\omega$ den vereinfachten Wert \glq 1\grq .
Daf"ur gelten folgende Bedingungen: $G$\footnote{Die Gravitationskonstante
G}=1, $m_{Erde}=1-c$, $m_{Mond}=c$ und $c \ll 1$.
\end{enumerate}
\subsection{Die Teile des Programms}
Um die Schwingungsbahnen eines Trojaners auf dem Computer zu simulieren benutzen wir
die Programmiersprache {\it Pascal}, da sich diese f"ur solche Berechnungen, die eine
iterativen Charakter besitzen bestens geeignet ist.

\subsubsection{Die Beschleunigung}
Die X-Komponente und die Y-Komponente der Kr"afte, die auf den Satelliten einwirken,
wird aus den gegebenen Gr"o\3en berechnet und wird unter Vernachl"assigung der Masse
des Satelliten dazu verwendet die X-Komponente und die Y-Komponente der auf den
Satelliten einwirkenden Beschleunigung zu errechnen. 

 \begin{verbatim}
	rM.x := Mond[Nr].x - Satellit.x; 
	rM.y := Mond[Nr].y - Satellit.y; 
	rE.x := Erde[Nr].x - Satellit.x; 
	rE.y := Erde[Nr].y - Satellit.y;

	rM2 := sqr(rM.x) + sqr(rM.y);
	rE2 := sqr(rE.x) + sqr(rE.y);
	rM3 := rM2 + sqrt(rM2);
	rE3 := rE2 + sqrt(rE2);
 \end{verbatim}

\subsubsection{Das Problem der Differentialgleichungen}
Um aus der oben hergeleiteten Berechnung der Beschleunigung und ihrer Richtung nun
den neuen Ort und die neue Geschwindigkeit des K"orpers zu errechnen mu\3 man eine
Differentialgleichung zweiter Ordnung zur Hilfe nehmen. Wir st"utzen uns hier auf die
Tatsache, da\3 die Beschleunigung $a$ die zweite Ableitung des Ortes $s$ darstellt.
Hierzu mu\3 man die X- und Y-Koordinate in zwei Arbeitsg"angen ausrechnen:
$$ x=f(x,x^{\prime\prime})$$
$$ y=f(y,y^{\prime\prime})\quad.$$
Eine genaue Berechnung der neuen X- und Y-Koordinate ist leider nicht m"oglich; es
l"a\3t sich nur eine Ann"ahrung an den wirklichen Wert machen.

Zur n"aherungsweisen Berechnung 